\documentclass[11pt]{article}

%\usepackage{listings}

%algoritmi
%\usepackage[compatibility=false]{caption}
%\usepackage{minted}

\usepackage{algorithm} 
\usepackage{algpseudocode}
\usepackage{listings}
\usepackage{xcolor}

\usepackage{comment}
\usepackage{subcaption}

\usepackage{hyperref}

%New colors defined below
\definecolor{codegreen}{rgb}{0,0.6,0}
\definecolor{codegray}{rgb}{0.5,0.5,0.5}
\definecolor{codepurple}{rgb}{0.58,0,0.82}
\definecolor{backcolour}{rgb}{0.95,0.95,0.92}

%Code listing style named "mystyle"
\lstdefinestyle{mystyle}{
  backgroundcolor=\color{backcolour}, commentstyle=\color{codegreen},
  keywordstyle=\color{magenta},
  numberstyle=\tiny\color{codegray},
  stringstyle=\color{codepurple},
  basicstyle=\ttfamily\footnotesize,
  breakatwhitespace=false,         
  breaklines=true,                 
  captionpos=b,                    
  keepspaces=true,                 
  numbers=left,                    
  numbersep=5pt,                  
  showspaces=false,                
  showstringspaces=false,
  showtabs=false,                  
  tabsize=2
}

%"mystyle" code listing set
\lstset{style=mystyle}


%form

\usepackage[utf8]{inputenc}
\usepackage[margin=1in]{geometry} 
\usepackage{amsmath,amsthm,amssymb,graphicx,mathtools,tikz,hyperref,multicol,cancel,enumitem,booktabs,float,pgfplots,multirow,mathrsfs,textcomp,gensymb,soul,changepage,threeparttable}
%\usepackage[table]{xcolor}
\usepackage[T1]{fontenc}
\usepackage[italian]{babel}
\usepackage{hyphenat}
\hyphenation{mate-mati-ca recu-perare}
\usetikzlibrary{positioning}
\pgfplotsset{compat=1.14}

\newcommand{\n}{\mathbb{N}}
\newcommand{\z}{\mathbb{Z}}
\newcommand{\q}{\mathbb{Q}}
\newcommand{\cx}{\mathbb{C}}
\newcommand{\real}{\mathbb{R}}
\newcommand{\field}{\mathbb{F}}
\newcommand{\ita}[1]{\textit{#1}}
\newcommand{\com}[2]{#1\backslash#2}
\newcommand{\oneton}{\{1,2,3,...,n\}}
\newcommand{\idea}[1]{\begin{gather*}#1\end{gather*}}
\newcommand{\ef}{\ita{f} }
\newcommand{\eff}{\ita{f}}
\newcommand{\proofs}[1]{\begin{proof}#1\end{proof}}
\newcommand{\inv}[1]{#1^{-1}}
\newcommand{\setb}[1]{\{#1\}}
\newcommand{\en}{\ita{n }}
\newcommand{\vbrack}[1]{\langle #1\rangle}
\newcommand{\qRa}{\quad \Rightarrow \quad}
\newcommand{\smaca}[1]{\textbf{\textsc{#1}}}

\newenvironment{theorem}[2][Teorema]{\begin{trivlist}
\item[\hskip \labelsep {\bfseries #1}\hskip \labelsep {\bfseries #2.}]}{\end{trivlist}}
\newenvironment{lemma}[2][Lemma]{\begin{trivlist}
\item[\hskip \labelsep {\bfseries #1}\hskip \labelsep {\bfseries #2.}]}{\end{trivlist}}
\newenvironment{exercise}[2][Esercizio]{\begin{trivlist}
\item[\hskip \labelsep {\bfseries #1}\hskip \labelsep {\bfseries #2.}]}{\end{trivlist}}
\newenvironment{proposition}[2][Proposizione]{\begin{trivlist}
\item[\hskip \labelsep {\bfseries #1}\hskip \labelsep {\bfseries #2.}]}{\end{trivlist}}
\newenvironment{corollary}[2][Corollario]{\begin{trivlist}
\item[\hskip \labelsep {\bfseries #1}\hskip \labelsep {\bfseries #2.}]}{\end{trivlist}}

\hypersetup {
    colorlinks,
    linkcolor=blue
}

\graphicspath{{img/}}

\begin{document}
\setlength{\parindent}{0pt}
\title{\vspace{-4em}{\large Roma Tre} \\
    Relazione Ingegneria dei Dati}
\author{Piergiorgio Fornaro (577925)}
\date{03/11/2025}
\maketitle

\vspace{-2em}\par\noindent\rule{\textwidth}{0.4pt}
\begin{center}
    {\Large\sc Indexer e Searcher su Lucene}
\end{center}
\par\noindent\rule{\textwidth}{0.4pt}

\hypersetup{colorlinks=true,linkcolor=blue,urlcolor=blue}

\section{Introduzione}
La seguente relazione ha lo scopo di rendicontare i risultati ottenuti in seguito alla creazione di codice per sfruttare Lucene come Indexer e Searcher su un piccolo database di dati.\newline
La versione di Lucene utilizzata è la 10.3.1, con JDK 25.0.1.\newline
Sono stati analizzati due campi principali per ogni documento: \textbf{Nome} e \textbf{Contenuto} ed è stata realizzata una GUI per poter interagire in maniera più agevole con il motore di ricerca.
\begin{comment}
Sono presenti due campi principali per ogni documento:
\begin{itemize}
    \item \textbf{Nome}: ovvero il nome del file. L'obiettivo della ricerca su questo campo è la corrispondenza esatta ma case insensitive. A tale scopo è stato utilizzato un TextField.
    \item \textbf{Contenuto}: ovvero ciò che è scritto dentro al file.
    L'obiettivo della ricerca su questo campo è trovare le parole cercate o che contengono quella corrispettiva stringa al loro interno. Quest'ultima cosa è resa possibile grazie all'aggiunta del carattere speciale \textbf{*} indicante "è presente qualsiasi cosa in questo punto". É stato utilizzato l'ItalianAnalyzer di Lucene, che gestisce la lingua italiana, rimuovendo le stopwords, uniformando i caratteri ed effettuando lo stemming (riduzione di una parola alla sua forma radice) per una ricerca più robusta.
\end{itemize}
\end{comment}
%É stata infine realizzata una GUI per poter interagire in maniera più agevole con il motore di ricerca.

\subsection{GitHub del progetto}
URL: \url{https://github.com/PiergiF/IngegneriaDeiDati_25-26_LuceneHomework2}


\section{Atricolazione del progetto}
Nella sua prima versione il progetto è scritto tutto in una classe per verificarne l'effettivo funzionamento. In futuro verrà Rifattorizzata in più classi e con l'aggiunta di test automatici e non manuali.\newline
Il codice è così articolato:
\begin{itemize}
    \item Prima parte contenente la \textbf{creazione} e l'inizializzazione della \textbf{GUI}, con tutti i pulsanti e la dark mode.
    \item Possibilità di \textbf{scelta della directory} da dove prendere i dati da indicizzare (ovviamente è presente una directory di default).
    \item Parte di \textbf{indicizzazione} in cui vengono visti i file nella directory dei dati selezionata. Per ogni file legge ed indicizza:
        \begin{itemize}
            \item \textbf{Nome}: ovvero il nome del file. L'obiettivo della ricerca su questo campo è la corrispondenza esatta ma case insensitive. A tale scopo è stato utilizzato un TextField.
            \item \textbf{Contenuto}: ovvero ciò che è scritto dentro al file. L'obiettivo della ricerca su questo campo è trovare le parole cercate o che contengono quella corrispettiva stringa al loro interno. Quest'ultima cosa è resa possibile grazie all'aggiunta del carattere speciale \textbf{*} indicante "è presente qualsiasi cosa in questo punto". É stato utilizzato l'ItalianAnalyzer di Lucene, che gestisce la lingua italiana, rimuovendo le stopwords, uniformando i caratteri ed effettuando lo stemming (riduzione di una parola alla sua forma radice) per una ricerca più robusta.
        \end{itemize}
        Se non è stato trovato nessun file, il problema viene segnalato all'utente. Si utilizza un IndexWriter per scrivere i documenti nell’indice. A fine indicizzazione viene riportato il tempo totale e, se selezionato, anche il tempo impiegato per ogni file. É poi possibile salvare i risultati in formato \textit{csv}. Si può ricreare l'inidice direttamente da programma, tramite apposito tasto che, se premuto, cancellerà il vecchio indice, creandone uno nuovo.
        \item Parte di \textbf{Ricerca}, che si occupa della ricerca dei documenti indicizzati. Apre l’indice salvato su disco e utilizza un IndexSearcher per interrogare i documenti. In base al campo selezionato sa su quale campo cercare e costruisce la Query tramite un QueryParser. La query ritorna ciò che si è cercato, il numero di occorrenze, l'eventuale file di riferimento (nel caso sia un contenuto) e lo score.
        \item Parte di \textbf{Esportazione} dei risultati ottenuti.
        \item Parte dedicata al \textbf{Log}.
        \item \textbf{Main} per far partire l'applicazione.
\end{itemize}


\section{Test}
I test sono stati momentaneamente realizzati a mano.\newline
Sono state effettuate query su 3 file \textit{txt}:
\begin{itemize}
    \item Roma.txt
    \item Lucene.txt
    \item Esercizi.txt
\end{itemize}
Query realizzate:
\begin{itemize}
    \item "Roma.txt" con ricerca sul nome. Risultato: Ricerca [nome]Roma.txt - Risultati: 1 - Roma.txt (score: 0,446)
    \item "Roma.json" con ricerca sul nome. Ricerca [nome]: Roma.json - Risultati: 0
    \item "Roma" con ricerca sul nome. Ricerca [nome]: Roma - Risultati: 0
    \item "ROMA.TXT" con ricerca sul nome. Ricerca [nome]: ROMA.TXT - Risultati: 1 - Roma.txt (score: 0,446)
    \item "Software" con ricerca sul contenuto.  Ricerca [contenuto]: Software - Risultati: 1 - lucene.txt (score: 0,455)
    \item "open source" con ricerca sul contenuto.  Ricerca [contenuto]: open source - Risultati: 1 - lucene.txt (score: 0,593)
    \item "opensource" con ricerca sul contenuto.  Ricerca [contenuto]: opensource - Risultati: 0
    \item "finire tutti gli esercizi" con ricerca sul contenuto. Ricerca [contenuto]: finire tutti gli esercizi - Risultati: 1 - esercizi.txt (score: 1,215)
    \item "esa*" con ricerca sul contenuto. Ricerca [contenuto]: esa* - Risultati: 1 - esercizi.txt (score: 1,000)
    \item "Paulo Dybala" con ricerca sul contenuto. Ricerca [contenuto]: Paulo Dybala - Risultati: 1 - Roma.txt (score: 1,170)
\end{itemize}

\subsection{Tempi di indicizzazione}
Sono stati indicizzati 3 file con i seguenti tempi:
\begin{itemize}
    \item Indicizzato: lucene.txt (2,1 ms)
    \item Indicizzato: Roma.txt (0,3 ms)
    \item Indicizzato: esercizi.txt (0,3 ms)
\end{itemize}
Indicizzazione completata in 0,025 secondi (3 file, 118,71 file/sec)

\section{Sviluppi futuri}
In futuro gli obiettivi sono di:
\begin{itemize}
    \item Rifattorizzare in più classi.
    \item Aggiungere test automatici.
    \item Aggiornare la relazione.
    \item Aggiungere nuove feature.
\end{itemize}
\end{document}

